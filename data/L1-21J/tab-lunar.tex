
%TTTTTTTTTTTTTTTTTTTTTTTTTTTTTTTTTTTTTTTTTTTTTTTT
%LUNAR SURFACE FEATURES TABLE
\begin{table*}
\centering
\begin{tabular}{cccc}
\hline\hline
Surface Feature & Selenographic Coordinates & Geocentric Position$^\dagger$ & Distance$^\dagger$ \\
                & (Lat.,Lon.)               & J2000 (RA,Dec)      & (km)     \\\hline
Moon Center & - & (8.16670,20.43896) & 356556\\
L1-21J & (-29.417,-67.897)$^\ddagger$ & (8.17988,20.25332) & 356056\\
Byrgius A & (-24.540,-63.830) & (8.18035,20.27369) & 355869\\
Grimaldi & (-5.530,-68.260) & (8.18398,20.35336) & 355644\\
Aristachus & (23.690,-47.490) & (8.18145,20.50518) & 355005\\
Plato & (51.640,-9.300) & (8.17206,20.64614) & 355155\\
Tycho & (-43.400,-11.260) & (8.16583,20.24611) & 355776\\
Copernicus & (9.640,-20.060) & (8.17367,20.46622) & 354512\\
Manilius & (14.440,9.060) & (8.16449,20.52050) & 354818\\
Dionysus & (2.770,17.290) & (8.16073,20.47551) & 355089\\
Chladni & (3.470,-0.230) & (8.16619,20.46314) & 354797\\
Kepler & (8.150,-37.990) & (8.17884,20.43986) & 354824\\
Bullialdus & (-20.750,-22.300) & (8.17146,20.32288) & 355229\\

\hline\hline
\multicolumn{4}{l}{\footnotesize $^\dagger$ Calculated geocentric coordinates and distance (see Section \ref{sec:parallax})}\\
\multicolumn{4}{l}{\footnotesize $^\ddagger$ Calculated with our geometrical procedure (see Section \ref{subsec:geolocation})}\\
\end{tabular}
\caption{Lunar features reference points and their selenographic coordinates, along with the apparent geocentric equatorial coordinate RA, Dec and geocentric distance results from parallax analysis.}
\label{tab:surfacefeatures}
\end{table*}
%TTTTTTTTTTTTTTTTTTTTTTTTTTTTTTTTTTTTTTTTTTTTTTTT
