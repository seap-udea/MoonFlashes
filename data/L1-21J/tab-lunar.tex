
%TTTTTTTTTTTTTTTTTTTTTTTTTTTTTTTTTTTTTTTTTTTTTTTT
%LUNAR SURFACE FEATURES TABLE
\begin{table*}
\centering
\begin{tabular}{cccc}
\hline\hline
Surface Feature & Selenographic Coordinates & Geocentric Position$^\dagger$ & Distance$^\dagger$ \\
                & (Lat.,Lon.)               & J2000 ($\alpha$, $\delta$)      & (km)     \\\hline
Moon Center & - & (8.1667,20.4362) & 357046\\
L1-21J & (-29.42,-67.90)$^\ddagger$ & (8.1799,20.2505) & 356553\\
Byrgius A & (-24.54,-63.83) & (8.1804,20.2709) & 356363\\
Grimaldi & (-5.53,-68.26) & (8.1840,20.3505) & 356135\\
Aristachus & (23.69,-47.49) & (8.1815,20.5024) & 355491\\
Plato & (51.64,-9.30) & (8.1721,20.6433) & 355636\\
Tycho & (-43.40,-11.26) & (8.1659,20.2433) & 356271\\
Copernicus & (9.64,-20.06) & (8.1737,20.4634) & 354999\\
Manilius & (14.44,9.06) & (8.1645,20.5177) & 355303\\
Dionysus & (2.77,17.29) & (8.1608,20.4727) & 355576\\
Chladni & (3.47,-0.23) & (8.1662,20.4603) & 355284\\
Kepler & (8.15,-37.99) & (8.1789,20.4370) & 355312\\
Bullialdus & (-20.75,-22.30) & (8.1715,20.3201) & 355721\\

\hline\hline
\multicolumn{4}{l}{\footnotesize $^\dagger$ Calculated geocentric coordinates and distance (see Section \ref{sec:parallax})}\\
\multicolumn{4}{l}{\footnotesize $^\ddagger$ Calculated with our geometrical procedure (see Section \ref{subsec:geolocation})}\\
\end{tabular}
\caption{Lunar features reference points and their selenographic coordinates, along with the apparent geocentric equatorial coordinate RA, Dec and geocentric distance results from parallax analysis.}
\label{tab:surfacefeatures}
\end{table*}
%TTTTTTTTTTTTTTTTTTTTTTTTTTTTTTTTTTTTTTTTTTTTTTTT
